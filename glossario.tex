
%**************************************************************
% Acronimi
%**************************************************************
\renewcommand{\acronymname}{Acronimi e abbreviazioni}

\newacronym[description={\glslink{aws}{Amazon Web Services}}]
    {aws}{AWS}{Amazon Web Services}

\newacronym[description={\glslink{apig}{Application Program Interface}}]
    {api}{API}{Application Program Interface}

\newacronym[description={Model-View-ViewModel}]
    {mvvm}{MVVM}{Model-View-ViewModel}

\newacronym[description={Internet of Things}]
    {iot}{IoT}{Internet of Things}

\newacronym[description={Database Management System}]
    {dbms}{DBMS}{Database Management System}

\newacronym[description={JavaScript Obbject Notation}]
    {json}{JSON}{JavaScript Obbject Notation}

\newacronym[description={Test-Driven Development}]
    {tdd}{TDD}{Test-Driven Development}

\newacronym[description={Information Technology}]
    {it}{IT}{Information Technology}

%**************************************************************
% Glossario
%**************************************************************
%\renewcommand{\glossaryname}{Glossario}

\newglossaryentry{apig}
{
    name=\glslink{api}{API},
    text=Application Program Interface,
    sort=api,
    description={in informatica con il termine \emph{\acrfull{api}} si indica ogni insieme di procedure disponibili al programmatore, di solito raggruppate a formare un set di strumenti specifici per l'espletamento di un determinato compito all'interno di un certo programma. La finalità è ottenere un'astrazione, di solito tra l'hardware e il programmatore o tra software a basso e quello ad alto livello semplificando così il lavoro di programmazione}
}

%\newglossaryentry{umlg}
%{
%    name=\glslink{uml}{UML},
%    text=UML,
%    sort=uml,
%    description={in ingegneria del software \emph{UML, Unified Modeling Language} (ing. linguaggio di modellazione unificato) è un linguaggio di modellazione e specifica basato sul paradigma object-oriented. L'\emph{UML} svolge un'importantissima funzione di ``lingua franca'' nella comunità della progettazione e programmazione a oggetti. Gran parte della letteratura di settore usa tale linguaggio per descrivere soluzioni analitiche e progettuali in modo sintetico e comprensibile a un vasto pubblico}
%}

\newglossaryentry{mock-up}
{
    name=Mock-Up,
    text=mock-up,
    sort=mock-up,
    description={Durante la progettazione un \emph{mock-up} è un modello utilizzato, ad esempio, per valutare la bontà della progettazione effettuata o per ottenere dei \emph{feedbacks} da parte dei clienti}
}

\newglossaryentry{backend}
{
    name=Backend,
    text=backend,
    sort=backend,
    description={Un'applicazione backend è un programma con il quale l'utente interagisce indirettamente, in generale attraverso l'utilizzo di un'applicazione \gls{frontend}. In una struttura client/server il backend è il server}
}

\newglossaryentry{open-source}
{
    name=Open-Source,
    text=open-source,
    sort=open-source,
    description={Per \emph{software open-source} si intende \emph{software} il cui codice sorgente è disponibile al pubblico, il quale può visionarlo e modificarlo. Il modello \emph{open-source} è un modello di sviluppo decentralizzato che incoraggia la collaborazione}
}

\newglossaryentry{frontend}
{
    name=Frontend,
    text=frontend,
    sort=frontend,
    description={Un'applicazione frontend è un programma con il quale l'utente interagisce direttamente. In una struttura client/server il frontend è il client.}
}

\newglossaryentry{stakeholder}
{
    name=Stakeholder,
    text=stakeholder,
    sort=stakeholder,
    description={Portatori di interesse. Nell’ambito del software, possono essere i clienti, fornitori, finanziatori e banche.}
}
