% !TEX encoding = UTF-8
% !TEX TS-program = pdflatex
% !TEX root = ../tesi.tex

%**************************************************************
\chapter{Lo stage}
\label{cap:stage}
%**************************************************************
\section{Rapporto dell'azienda}
Puntando molto sui giovani, Wavelop è sempre più presente all'evento \textbf{Stage-IT} organizzato da Assindustria Venetocentro in collaborazione con l'Università degli Studi di Padova. \
Questo evento consente agli studenti di mettersi in gioco entrando in contatto con il mondo del lavoro e mettendo in pratica tutto ciò che hanno appreso durante il percoso di studi, mentre le \
aziende, in questo modo, introducono nel loro organico personale giovane in grado di creare valore aggiunto al loro interno.
In Wavelop, essendo una realtà piccola e recente lo \emph{stage} non è visto solo come un aumento di personale gratuito, ma come un'opportunità di una possibile collaborazione con lo \
stagista; infatti, in ottica di una futura assunzione, lo \emph{stage} è il primo \emph{step} nel processo di assunzione dell'azienda. L'ambiente giovane e informale dell'azienda ha fatto \
si che il mio inserimento nel \emph{team} sia avvenuto velocemente, instaurando sin da subito un rapporto amichevole con i colleghi; avendo svolto principalmente lo \
\emph{stage} da remoto, il rapporto instaurato con i colleghi ha permesso di farmi sentire parte integrante del \emph{team} nonostante la mia assenza in ufficio.

%**************************************************************
\section{Il progetto}
Il progetto che mi è stato assegnato dall'azienda riguarda l'analisi e lo sviluppo di un'applicazione \
\emph{web} con lo scopo di gestire gli ordini tra l'azienda Lago e i suoi fornitori. \ 
Questo progetto è stato commissionato a Wavelop dall'azienda Lago, la quale fino ad oggi gestiva i suoi ordini attraverso \
l'uso del \emph{software Google Sheets}. A causa di una riduzione del limite mensile di chiamate alle \emph{\acrshort{api}} \
di \emph{Google}, ha deciso di abbandonare la soluzione corrente e di procedere con la realizzazione di un \
\emph{software} indipendente da \emph{Google Sheets}. \\

Le funzionalità richieste da Lago che Wavelop si impegna ad implementare sono le seguenti: \
\begin{itemize}
  \item la possibilità di bloccare le colonne e impedirne lo \emph{scroll};
  \item la modifica del valore di una cella;
  \item l'eliminazione di una riga, se l'utente ha i requisiti necessari;
  \item la modifica di più celle appartenenti ad una stessa colonna mediante \emph{drag and drop};
  \item la ricerca mediante l'implementazione di filtri appositi.
\end{itemize}
A queste funzionalità di base si aggiungono quelle per la gestione degli utenti e per il caricamento e visualizzazione \
di \emph{file} allegati. 

%**************************************************************
\section{Obiettivi}

\subsection{Obiettivi dello stage}
Nella tabella sottostante sono elencati gli obiettivi prefissati per lo svolgimento dello \emph{stage}, elaborati durante \
i primi incontri con il mio \emph{tutor} aziendale. \
Sono identificabili tramite una lettera e un numero progressivo; la lettera indica:
\begin{itemize}
  \item \textbf{O} per i requisiti obbligatori, vincolanti in quanto obiettivo primario richiesto dal committente;
  \item \textbf{D} per i requisiti desiderabili, non vincolanti o strettamente necessari, ma dal riconoscibile valore aggiunto.
\end{itemize}

\begin{center}
  \begin{table}[h]
    \begin{tabular}{ c | p{\textwidth - 40pt} }
      \textbf{O01} & Apprendimento delle tecnologie di sviluppo come \emph{React}, \emph{NodeJS}, \emph{MongoDB} e \emph{Lerna} \\ \hline
      \textbf{O02} & Apprendimento della tecnologia di versionamento \emph{Git} \\ \hline
      \textbf{O03} & Apprendimento delle tecnologie di \emph{deployment} \emph{Docker} e \emph{docker-compose} \\ \hline
      \textbf{O04} & Implementazione delle funzionalità di autenticazione \\ \hline
      \textbf{O05} & Implementazione delle funzionalità relative alla lista dei fornitori \\ \hline
      \textbf{O06} & Implementazione delle funzionalità relative alla griglia di un fornitore \\ \hline
      \textbf{D01} & Implementazione delle funzionalità relative al caricamento e visualizzazione degli allegati di un ordine \\ \hline
      \textbf{D02} & Implementazione delle funzionalità relative alla gestione degli utenti \\
    \end{tabular}
    \caption{Prospetto degli obiettivi prefissati}
  \end{table}
\end{center}

\subsection{Obiettivi personali}
Il mio obiettivo durante la ricerca del progetto di \emph{stage} non era di trovare solo un'impresa che mi desse la possibilità di adempiere ad un obbligo scolastico, \
ma di cercare un'azienda in grado di fornirmi progetti interessanti utili a creare le basi per uno sviluppo professionale futuro. Allo stesso tempo cercavo un'azienda che \
mi permettesse di imparare nuove tecnologie e di migliorare l'uso di quelle già conosciute. L'azienda che mi ha permesso di raggiungere tutti i miei obiettivi sono riuscito a \
trovarla grazie all'evento \textbf{Stage-IT} durante il quale ho effettuato diversi colloqui con le aziende partecipanti; fra tutte, quella che mi ha incuriosito di più, è stata Wavelop \
sia per l'ambiente di lavoro giovane sia per la tipologia di progetti proposti. Nei giorni successivi all'evento sono stato contattato da Matteo Granzotto, colui che sarebbe stato il \
mio \emph{tutor} durante il periodo di \emph{stage}, con il quale ho svolto un colloquio conoscitivo, al termine del quale mi ha assegnato un test tecnico da svolgere su una piattaforma \
online. Superato questo test ho deciso di scegliere questa azienda sia per l'interessante progetto proposto ma anche per lo \emph{stack} tecnologico utilizzato dall'azienda il quale è composto \
da tecnologie a me in gran parte sconosciute. Inoltre ha influito positivamente la forte propensione al \emph{remote working} da parte dell'azienda, che mi ha permesso di svolgere lo \emph{stage} \
senza troppe spese per raggiungere l'ufficio.

%**************************************************************
\section{Vincoli}

\subsection{Vincoli tecnologici}
Il linguaggio richiesto dall'azienda per la realizzazione del progetto di \emph{stage} è \emph{\textbf{TypeScript}}: \
un linguaggio basato su \emph{\textbf{JavaScript}} al quale aggiunge il controllo statico dei tipi diminuendo così la \
probabilità di eventuali \emph{bugs}. È il linguaggio utilizzato dall'azienda sia lato \emph{\gls{frontend}} che lato \
\emph{\gls{backend}}. Nonostante \emph{\textbf{JavaScript}} sia un linguaggio nato per lo sviluppo \emph{frontend}, \
l'esecuzione di \emph{\textbf{TypeScript}} lato \emph{backend} è resa possibile grazie all'utilizzo di \
\emph{\textbf{NodeJS}}. \emph{\textbf{NodeJS}} è un \emph{runtime \textbf{JavaScript}} con architettura orientata agli \
eventi e permette l'esecuzione asincrona delle istruzioni; il vantaggio principale nell'utilizzo di questa tecnologia è \
la sua scalabilità infatti è ampiamente utilizzata per l'implementazione di applicazioni di rete. Per l'implementazione del \
\emph{frontend}, invece, è stato utilizzato \emph{\textbf{React}}: una libreria per la realizzazione di interfacce utente \
aderente al paradigma dichiarativo. Con questa libreria le interfacce utente sono realizzate mediate due tipi di componenti: \
\begin{itemize}
  \item \emph{Function components} che vengono dichirati mediante funzioni;
  \item \emph{Class-based components} che vengono dichiarati mediante classi.
\end{itemize}
Per quanto riguarda l'implementazione di \emph{unit tests} l'azienda ha richiesto l'utilizzo di \emph{\textbf{Jest}}: \
il framework di \emph{testing} per il linguaggio \emph{\textbf{JavaScript}} più diffuso e che permette facilmente di \
fare il \emph{mocking} di funzioni e di generare le informazioni di \emph{code coverage}, le quali permettono di \
valutare la qualità del codice prodotto.
Per il salvataggio dei dati in modo persistente è stato scelto \emph{\textbf{MongoDB}}. Questo \emph{\acrshort{dbms}} \
è la soluzione ideale per il progetto che mi è stato assegnato in quanto memorizza i \emph{record}, chiamati \emph{documents}, \
in formato \emph{\acrshort{json}} e, utilizzando anche all'interno dell'applicazione questo formato di rappresentazione \
dei dati, mi ha permesso di poter utilizzare direttamente i dati risultato della \emph{query} senza la necessità di ulteriori \
conversioni. Un'altra caratteristica importante è il fatto che si basa su una struttura \emph{schemaless}, \
ovvero, più \emph{records} possono avere una struttura differente; questo è fondamentale nel contesto dell'applicazione commissionata \
da Lago in quanto è richiesto che, durante la configurazione di un fornitore, vengano visualizzati campi specifici.

\subsection{Vincoli temporali}
L’Università di Padova stabilisce che la durata dello \emph{stage} sia compresa tra le 300 e le 320 ore.
Il carico di lavoro è stato suddiviso in otto settimane consecutive, con un impegno lavorativo di 40 ore settimanali.
Prima dell’inizio dello \emph{stage} formativo è stato redatto, consensualmente con l'azienda, un Piano di Lavoro dove \
sono stati definiti gli obiettivi dello \emph{stage} ripartiti tra le varie settimane.

%**************************************************************
\section{Pianificazione}
Lo \emph{stage} è iniziato il 04/10/2021 ed è terminato il 26/11/2021. \\
Le attività previste per la realizzazione del progetto sono state suddivise settimanalmente come segue:
\begin{itemize}
  \item \textbf{Prima settimana:} 
    \begin{itemize}
      \item Introduzione al \emph{way of working} dell'azienda;
      \item Studio delle tecnologie impiegate nel progetto;
      \item Realizzazione di uno \emph{spike \gls{frontend}} e uno \emph{\gls{backend}} per verificare quanto imparato.
    \end{itemize}
  \item \textbf{Seconda settimana:} svolgimento dell'analisi dei requisiti e inizio dell'analisi architetturale;
  \item \textbf{Terza settimana:} conclusione dell'analisi architetturale e sviluppo tramite la metodologia del \acrshort{tdd} delle storie assegnate;
  \item \textbf{Quarta settimana:} 
    \begin{itemize}
      \item Sviluppo tramite la metodologia del \acrshort{tdd} delle \emph{user stories} assegnate;
      \item \emph{Sprint review} con il referente;
      \item Analisi e definizione delle \emph{user stories} per il \emph{backlog} dello \emph{Sprint} successivo assieme \
      al referente.
    \end{itemize}
  \item \textbf{Quinta settimana:} 
    \begin{itemize}
      \item Sviluppo tramite la metodologia del \acrshort{tdd} delle \emph{user stories} assegnate;
      \item \emph{Sprint review} con il referente.
    \end{itemize}
  \item \textbf{Sesta settimana:} 
    \begin{itemize}
      \item Sviluppo tramite la metodologia del \acrshort{tdd} delle \emph{user stories} assegnate;
      \item \emph{Sprint review} con il referente;
      \item Analisi e definizione delle \emph{user stories} per il \emph{backlog} dello \emph{Sprint} successivo assieme \
      al referente.
    \end{itemize}
  \item \textbf{Settima settimana:} 
    \begin{itemize}
      \item Sviluppo tramite la metodologia del \acrshort{tdd} delle \emph{user stories} assegnate;
      \item \emph{Sprint review} con il referente.
    \end{itemize}
  \item \textbf{Ottava settimana:} 
    \begin{itemize}
      \item Sviluppo tramite la metodologia del \acrshort{tdd} delle \emph{user stories} assegnate;
      \item \emph{Sprint review} con il referente;
      \item Collaudo finale.
    \end{itemize}
\end{itemize}