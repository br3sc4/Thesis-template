% !TEX encoding = UTF-8
% !TEX TS-program = pdflatex
% !TEX root = ../tesi.tex

%**************************************************************
\chapter{Lo stage}
\label{cap:stage}
%**************************************************************
\section{Rapporto dell'azienda}
In questa sezione descrivo il rapporto dell'azienda nei confronti degli stage.

%**************************************************************
\section{Il progetto}
Il progetto che mi è stato asseganto dall'azienda riguarda l'analisi e lo sviluppo di un'applicazione \
\emph{web} con lo scopo di gestire gli ordini tra l'azienda Lago e i suoi fornitori. \ 
Questo progetto è stato commissionato a Wavelop dall'azienda Lago, la quale fino ad oggi gestiva i suoi ordini attraverso \
l'uso del \emph{software Google Sheets}. A causa di una riduzione del limite mensile di chiamate alle \emph{\acrshort{api}} \
di \emph{Google}, ha deciso di abbandonare la soluzione corrente e di procedere con la realizzazione di un \
\emph{software} indipendente da \emph{Google Sheets}. \\

Le funzionalità richieste da Lago che Wavelop si impegna ad implementare sono le seguenti:
\begin{itemize}
  \item la possibilità di bloccare le colonne e impedirne lo \emph{scroll};
  \item la modifica del valore di una cella;
  \item l'eliminazione di una riga, se l'utente ha i requisiti necessari;
  \item la modifica di più celle appartenenti ad una stessa colonna mediante \emph{drag and drop};
  \item la ricerca mediante l'implemntazione di filtri appositi.
\end{itemize}
A queste funzionalità di base si aggiungono quelle per la gestione degli utenti e per il caricamento e visualizzazione \
di \emph{file} allegati. 

%**************************************************************
\section{Obiettivi}

\subsection{Obiettivi dello stage}
Nella sottostante sono elencati gli obiettivi prefissati per lo svolgimento dello \emph{stage}, elaborati durante \
i primi incontri con il mio \emph{tutor} aziendale. \
Sono identificabili tramite una lettera e un numero progressivo; la lettera indica:
\begin{itemize}
  \item \textbf{O} per i requisiti obbligatori, vincolanti in quanto obiettivo primario richiesto dal committente;
  \item \textbf{D} per i requisiti desiderabili, non vincolanti o strettamente necessari, ma dal riconoscibile valore aggiunto.
\end{itemize}

\begin{center}
  \begin{table}[h]
    \begin{tabular}{ c | p{\textwidth - 40pt} }
      \textbf{O01} & Apprendimento delle tecnologie di sviluppo come \emph{React.js}, \emph{NodeJs}, \emph{MongoDB} e \emph{Lerna} \\ \hline
      \textbf{O02} & Apprendimento della tecnologia di versionamento \emph{Git} \\ \hline
      \textbf{O03} & Apprendimento delle tecnologie di \emph{deployment} \emph{Docker} e \emph{docker-compose} \\ \hline
      \textbf{O04} & Implementazione delle funzionalità di autenticazione \\ \hline
      \textbf{O05} & Implementazione delle funzionalità realitive alla lista dei fornitori \\ \hline
      \textbf{O06} & Implementazione delle funzionalità relative alla griglia di un fornitore \\ \hline
      \textbf{D01} & Implementazione delle funzionalità relative al caricamento e visualizzazione degli allegati di un ordine \\ \hline
      \textbf{D02} & Implementazione delle funzionalità relative alla gestione degli utenti \\
    \end{tabular}
    \caption{Prospetto degli obiettivi prefissati}
    \label{table:2.1}
  \end{table}
\end{center}

\subsection{Obiettivi personali}
In questa sezione descrivo gli obiettivi personali che mi hanno spinto a scegliere questo progetto.

%**************************************************************
\section{Vincoli}

\subsection{Vincoli di dominio}
In questa sezione descrivo i vincoli di dominio del progetto.

\subsection{Vincoli tecnologici}
In questa sezione descrivo i vincoli tecnologici del progetto.

\subsection{Vincoli metodologici}
In questa sezione descrivo i vincoli metodologici del progetto.

\subsection{Vincoli temporali}
In questa sezione descrivo i vincoli temporali del progetto.

%**************************************************************
\section{Pianificazione}
In questa sezione descrivo la pianificazione del progetto di stage diviso per settimane.