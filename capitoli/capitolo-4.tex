% !TEX encoding = UTF-8
% !TEX TS-program = pdflatex
% !TEX root = ../tesi.tex

%**************************************************************
\chapter{Valutazione retrospettiva}
\label{cap:valutazione}
%**************************************************************
\section{Resoconto degli obbiettivi aziendali raggiunti}

\subsection{Funzionalità dell'applicazione}
Al termine del periodo di \emph{stage} posso affermare di avere soddisfatto tutti gli obiettivi definiti dall'azienda prima di inziare il progetto.
Ho implementato tutte le funzionalità ritenute obbligatorie per la prima \emph{release} dell'applicazione, ovvero:
\begin{itemize}
  \item l'autenticazione degli utenti mediante \emph{\textbf{Google SSO}};
  \item la ricerca degli ordini mediante filtri;
  \item la modifica singola di un campo di un ordine;
  \item la modifica multipla mediante \emph{drag-and-drop} di un campo di più ordini;
  \item il blocco di colonne e impedirne lo \emph{scroll} orizzontale;
  \item l'eliminazione di uno o più ordini, se l'utente è in possesso dei permessi necessari.
\end{itemize}

Ho implementato, inoltre, le funzionalità ritenute facoltative, ovvero:
\begin{itemize}
  \item la gestione degli utenti;
  \item il caricamento e visualizzazione dei file allegati agli ordini.
\end{itemize}

Ciò è stato possibile grazie allo svolgimento di tutte le attività specificate nelle 16 \emph{user stories}, 12 delle quali obbligatorie e 8 facoltative. 

\subsection{Formazione}
Grazie a questa esperienza di \emph{stage} ho avuto la possibilità di imparare nozioni nuove e di approfondirne altre.
Il mio bagaglio tecnologico si è arricchito delle seguenti tecnologie:
\begin{itemize}
  \item \textbf{\emph{NodeJS}:} questa tecnologia l'avevo già usata in passato ma, grazie allo \emph{stage}, ho potuto apprendere quali sono le \emph{best practices} e approfondire le conoscenze a riguardo. Ho notato, inoltre, una diversa metodologia di lavoro rispetto a quella usata durante le mie prime esperienze;
  \item \textbf{\emph{MongoDB}:} di questa tecnologia, prima dello \emph{stage}, ne avevo solamente sentito parlare e mi ha sempre incuriosito. Durante questo periodo ho avuto la possibilità di avvicinarmi al mondo dei \emph{database} non relazionali, imparando il loro funzionamento e le regole per una buona progettazione;
  \item \textbf{\emph{React}:} lo studio approfondito di questa tecnologia mi ha permesso di imparare le nozioni di \emph{hooks} e di \emph{context}, oltre a come implementare correttamente i componenti delle pagine. L'utilizzo degli \emph{hooks} mi ha permesso di creare componenti di tipo \emph{stateful} senza l'utilizzo di classi, ottenendo così, codice più pulito e meglio organizzato.
\end{itemize}

Oltre alle tecnologie, ho appreso nuove metodologie di lavoro:
\begin{itemize}
  \item \textbf{Metodologia Agile Scrum:} non essendo mai entrato in contatto con un contesto lavorativo, le mie conoscenze riguardanti il \emph{project management} erano legate esclusivamente al progetto di Ingegneria del \emph{Software}, durante il quale ho utilizzato un modello di sviluppo incrementale. Quest'ultimo, a differenza delle metodologie Agili, è più rigido, infatti i requisiti vengono assegnati ai diversi incrementi solo a progettazione finita, senza la possibilità di modifiche successive. Con le metodologie Agili, invece, ci si svincola dall'eccessiva rigidità, in quanto le \emph{user stories} da svolgere in uno \emph{sprint}, vengono scelte in base alle priorità dell'azienda. Grazie allo \emph{stage} ho appreso questa nuova metodologia di lavoro e, grazie alla sua applicazione, ho compreso meglio le sue dinamiche;
  \item \textbf{\emph{\acrlong{tdd}}:} questo modello di sviluppo del \emph{software} mi era stato introdotto durante il percorso di studio universitario ma, purtroppo, non avevo avuto la possibilità di applicarlo. Durante lo \emph{stage} non ho solo avuto la possibilità di applicarlo, fissando meglio i concetti visti in ateneo, ma anche la possibilità di approfondire le mie conoscenze a riguardo, in modo da applicarlo nel migliore dei modi. Mediante questo metodo di sviluppo ho imparato a pensare al \emph{software} in termini di piccole unità che possono essere testate in modo indipendente. Scrivendo i \emph{test} prima del codice, ne vengono implementati in grande quantità, i quali ci assicurano che il \emph{software} si comporterà sempre alla stessa maniera, anche dopo eventuali cambiamenti durante l'attività di \emph{refactoring}. L'implementazione di un gran numero di \emph{test}, inoltre, mi ha permesso di stimare in maniera più precisa lo stato di avanzamento del progetto.
\end{itemize}

%**************************************************************
\section{Bilancio dell'esperienza di \emph{stage}}

\subsection{Obiettivi personali raggiunti}
Oltre ad avere raggiunto tutti gli obiettivi aziendali, ho raggiunto gran parte degli obiettivi personali prefissati durante la scelta del progetto.
Il progetto, svolto durante questo periodo, mi ha permesso di approfondire alcune tecnologie che conoscevo e, inoltre, mi ha permesso di impararne molte altre, ciò mi ha messo nella condizione di ampliare molto il mio bagaglio di conoscenze tecniche. 
Ho raggiunto, inoltre, l'obiettivo di imparare le metodologie Agili, ampliando così le mie conoscenze relative al \emph{project management}.
L'unico obiettivo che, purtroppo, non ho soddisfatto è quello di lavorare a progetti innovativi, in quanto, il prodotto implementato non sfrutta tecnologie innovative, per citarne alcune: l'intelligenza artificiale, l'\emph{\acrlong{iot}} o l'architettura \emph{serverless}.
Grazie al buon lavoro svolto, al termine dello \emph{stage}, mi è stata offerta la possibilità di entrare a far parte del \emph{team} di Wavelop, raggiungendo così un altro obiettivo che mi ero prefissato.
In questo modo, ho la possibilità di raggiungere anche l'unico obiettivo che non sono riuscito a soddisfare durante in questa esperienza.

\subsection{Competenze acquisite}
L'esperienza di \emph{stage} svolta presso Wavelop ha avuto un ruolo molto importante per la mia crescita professionale.
Grazie all'ambiente giovane e informale sono riuscito ad integrarmi al meglio all'interno del \emph{team} di sviluppo, ciò mi ha permesso di svolgere il lavoro nelle condizioni migliori e di migliore le capacità di ascolto e interazione nei confronti degli altri componenti del \emph{team}.
Inoltre, ho potuto ampliare le conoscenze nell'ambito del \emph{project management}, ma soprattutto, ho migliorato le mie capacità di analisi del problema.
Per quanto riguardo le abilità di \emph{design}, ero già in possesso di buone conoscenze di base, apprese durante il corso di Ingegneria del \emph{Software} ma, mediante lo \emph{stage}, sono riuscito a consolidarle e ad affinarle, capendo quando è meglio utilizzare un \emph{design pattern} piuttosto che un altro e, soprattutto, ho appreso la loro applicazione nell'ambito delle \emph{web applications}.
Tutto ciò è stato reso possibile grazie allo studio di articolo tecnici e, soprattutto, ai confronti con il mio \emph{tutor} aziendale, che si è reso sempre disponibile per dei confronti.
Oltre alla crescita dal punto di vista professionale, questa esperienza mi ha fatto cresce anche come persona, migliorando le mie capacità di comunicazione, nonostante sia una persona introversa.

\subsection{Competenze mancanti in sede di inizio \emph{stage}}
Molte delle tecnologie di cui ho fatto uso durante lo \emph{stage} erano già presenti nel mio bagaglio di conoscenze, sia perché le ho utilizzate durante il corso di Ingegneria del \emph{Software}, ma anche perché le ho usate per attività personali e \emph{side projects}.
Nonostante questo, le mie competenze non potevano considerarsi complete, perché molti temi richiedevano conoscenze che non erano trattate in ambito universitario.
Per questo motivo ho dovuto dedicare del tempo allo studio all'approfondimento delle tecnologie per poterle utilizzare nello svolgimento delle attività aziendali.
La competenza principale che mi è mancata durante lo \emph{stage}, è la conoscenza dei \emph{database NoSQL}.
In ambito universitario, infatti, vengono studiati solamente i \emph{database SQL}, ovvero quelli relazionali, mentre quelli \emph{NoSQL} non vengono neanche accennati.
La loro introduzione in ambito universitario, sarebbe un grande valore aggiunto per il corso di studi, in quanto, al giorno d'oggi, il loro utilizzo è sempre più diffuso.

%**************************************************************
\section{Considerazioni finali}
Anche se alcune tecnologie non hanno avuto sufficiente spazio nel corso di studi, l'università mi ha fornito le metodologie per affrontare la continua evoluzione tecnologica nel modo giusto.
Durante i vari colloqui, per la scelta del progetto, ho notato una certa distanza tra le tecnologie studiate e quelle richieste nel mondo del lavoro, però, grazie alla metodologia fornitami dal corso di Laurea, non ho avuto problemi ad adattarmi ad esse.
Il corso di Laurea in Informatica riesce, quindi, a fornire delle metodologie molto valide che riescono a plasmare in modo corretto lo studente e questo è l'aspetto più importante. 
Tra tutti gli insegnamenti ho apprezzato di più quelli ottenuti durante il corso di Ingegneria del \emph{Software}, in quanto mi hanno aiutato molto nell'aspetto metodologico dell'affrontare le sfide.