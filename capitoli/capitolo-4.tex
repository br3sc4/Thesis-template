% !TEX encoding = UTF-8
% !TEX TS-program = pdflatex
% !TEX root = ../tesi.tex

%**************************************************************
\chapter{Valutazione retrospettiva}
\label{cap:valutazione}
%**************************************************************
\section{Resoconto degli obbiettivi aziendali raggiunti}

\subsection{Funzionalità dell'applicazione}
Al termine del periodo di \emph{stage} posso affermare di avere soddisfatto tutti gli obiettivi definiti dall'azienda prima di inziare il progetto.
Ho implementato tutte le funzionalità ritenute obbligatorie per la prima \emph{release} dell'applicazione, ovvero:
\begin{itemize}
  \item l'autenticazione degli utenti mediante \emph{\textbf{Google SSO}};
  \item la ricerca degli ordini mediante filtri;
  \item la modifica singola di un campo di un ordine;
  \item la modifica multipla mediante \emph{drag-and-drop} di un campo di più ordini;
  \item il blocco di colonne e impedirne lo \emph{scroll} orizzontale;
  \item l'eliminazione di uno o più ordini, se l'utente è in possesso dei permessi necessari.
\end{itemize}

Ho implementato, inoltre, le funzionalità ritenute desiderabili da parte dell'azienda, ovvero:
\begin{itemize}
  \item la gestione degli utenti;
  \item il caricamento e visualizzazione dei file allegati agli ordini.
\end{itemize}

Ciò è stato possibile grazie allo svolgimento di tutte le attività specificate nelle 16 \emph{user stories}, 12 delle quali obbligatorie e 8 facoltative. 

\subsection{Formazione}
Grazie a questa esperienza di \emph{stage} ho avuto la possibilità di imparare nozioni nuove e di approfondirne altre.
Il mio bagaglio tecnologico si è arricchito delle seguenti tecnologie:
\begin{itemize}
  \item \textbf{\emph{NodeJS}:} questa tecnologia l'avevo già usata in passato ma, grazie allo \emph{stage}, ho potuto apprendere quali sono le \emph{best practices} e approfondire le conoscenze a riguardo. Ho notato, inoltre, una diversa metodologia di approccio rispetto a quella usata durante le mie prime esperienze;
  \item \textbf{\emph{MongoDB}:} di questa tecnologia, prima dello \emph{stage}, ne avevo solamente sentito parlare e mi ha sempre incuriosito. Durante questo periodo ho avuto la possibilità di avvicinarmi al mondo dei \emph{database} non relazionali, imparando il loro funzionamento e le regole per una buona progettazione;
  \item \textbf{\emph{React}:} lo studio approfondito di questa tecnologia mi ha permesso di imparare le nozioni di \emph{hooks} e di \emph{context}, oltre a come implementare correttamente i componenti delle pagine. L'utilizzo degli \emph{hooks} mi ha permesso di creare componenti di tipo \emph{stateful} senza l'utilizzo di classi, ottenendo così, codice più pulito e meglio organizzato.
\end{itemize}

Oltre alle tecnologie, ho appreso nuove metodologie di lavoro:
\begin{itemize}
  \item \textbf{Metodologia Agile \emph{Scrum}:} non essendo mai entrato in contatto con un contesto lavorativo, le mie conoscenze riguardanti il \emph{project management} erano legate esclusivamente al progetto di Ingegneria del \emph{Software}, durante il quale ho utilizzato un approccio completamente diverso. Grazie allo \emph{stage} ho appreso un nuovo metodo di lavoro e, grazie alla sua applicazione, ho compreso meglio le sue dinamiche;
  \item \textbf{\emph{\acrlong{tdd}}:} questo modello di sviluppo del \emph{software} mi era stato introdotto durante il percorso di studio universitario ma, purtroppo, non avevo avuto la possibilità di applicarlo. Durante lo \emph{stage} non ho solo avuto la possibilità di applicarlo, fissando meglio i concetti visti in ateneo, ma anche la possibilità di approfondire le mie conoscenze a riguardo, in modo da applicarlo nel migliore dei modi.
\end{itemize}

%**************************************************************
\section{Bilancio formativo}

\subsection{Competenze acquisite}
In questa sezione descrivo le competenze acquisite durante lo stage.

\subsection{Competenze mancanti in sede di inizio stage}
In questa sezione descrivo le competenze mancanti ad inisio stage.

\subsection{Preparazione accademica}
In questa sezione esprimo le considerazioni su quanto offerto dal piano di studi per affrontare il mondo del lavoro.