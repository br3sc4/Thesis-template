% !TEX encoding = UTF-8
% !TEX TS-program = pdflatex
% !TEX root = ../tesi.tex

%**************************************************************
\chapter{Valutazione retrospettiva}
\label{cap:valutazione}
%**************************************************************
\section{Analisi dei risultati}

\subsection{Raggiungimento degli obiettivi aziendali}
Al termine del periodo di \emph{stage} posso affermare di avere soddisfatto tutti gli obiettivi definiti dall'azienda prima di inziare il progetto.
Ho implementato tutte le funzionalità ritenute obbligatorie per la prima \emph{release} dell'applicazione, ovvero:
\begin{itemize}
  \item l'autenticazione degli utenti mediante \emph{\textbf{Google SSO}};
  \item la ricerca degli ordini mediante filtri;
  \item la modifica singola di un campo di un ordine;
  \item la modifica multipla mediante \emph{drag-and-drop} di un campo di più ordini;
  \item il blocco di colonne e impedirne lo \emph{scroll} orizzontale;
  \item l'eliminazione di uno o più ordini, se l'utente è in possesso dei permessi necessari.
\end{itemize}

Ho implementato, inoltre, le funzionalità ritenute desiderabili da parte dell'azienda, ovvero:
\begin{itemize}
  \item la gestione degli utenti;
  \item il caricamento e visualizzazione dei file allegati agli ordini.
\end{itemize}

Ciò è stato possibile grazie allo svolgimento di tutte le attività specificate nelle 16 \emph{user stories}, 12 delle quali obbligatorie e 8 facoltative. 

\subsection{Problematiche riscontrate}
In questa sezione elenco e descrivo le problematiche riscontrate durante lo stage.

%**************************************************************
\section{Bilancio formativo}

\subsection{Competenze acquisite}
In questa sezione descrivo le competenze acquisite durante lo stage.

\subsection{Competenze mancanti in sede di inizio stage}
In questa sezione descrivo le competenze mancanti ad inisio stage.

\subsection{Preparazione accademica}
In questa sezione esprimo le considerazioni su quanto offerto dal piano di studi per affrontare il mondo del lavoro.