% !TEX encoding = UTF-8
% !TEX TS-program = pdflatex
% !TEX root = ../tesi.tex

%**************************************************************
\chapter{Implementazione del progetto}
\label{cap:implementazione}
%**************************************************************
\section{Pianificazione}
In questa sezione descrivo la pianificazione del progetto di satge diviso per settimane.

%**************************************************************
\section{Analisi dei requisiti}

\subsection{Attori}
In questa sezione descrivo gli attori che interagiranno con l'applicazione.

\subsection{\emph{User Stories}}
In questa sezione elenco le \emph{user stories} individuate.

\subsection{Requisiti}
In questa sezione elenco i requisiti individuati.

\subsection{Tracciamento dei requisiti}
In questa sezione realizzo una tabella per tracciare i requisiti ripetto le \emph{user sories}.

%**************************************************************
\section{Progettazione architetturale}

\subsection{Introduzione}
In questa sezione introduco l'architettura \emph{client-server}.

\subsection{Archtettura del \emph{backend}}
In questa sezione illustro l'architettura del \emph{backend}.

\subsection{Archtettura del \emph{frontend}}
In questa sezione illustro l'architettura del \emph{frontend}.

\subsection{Struttura del \emph{database}}
In questa sezione illustro la struttura del \emph{databse}.

%**************************************************************
\section{Progettazione di dettaglio}

\subsection{API per l'autenticazione}
In questa sezione descrivo le API necessarie per l'autenticazione di un utente.

\subsection{API per la manipolazione dei dati della griglia}
In questa sezione descrivo le API per la visualizzazione, la modifica e l'eliminazione dei dati della griglia.

\subsection{Diagrammi di sequenza}
In questa sezione illustro i diagrammi dii sequenza realativi all'autencazione e alla manipolazione dei dati della griglia.

%**************************************************************
\section{Sviluppo}

\subsection{Autenticazione}
In questa sezione illustro la porzione di codice necessaria per l'autenticazione.

\subsection{Griglia del fornitore}
In questa sezione illustro le porzioni di codice necessarie per la realizzazione della griglia del fornitore.

%**************************************************************
\section{Verifica e validazione}

\subsection{Analisi statica}
In questa sezione descrivo gli strumenti utlizzate per l'analisi statica.

\subsection{\emph{Unit Test}}
In questa sezione descrivo il framwork Jest e illustro i risultati dei test effettuati con la \emph{feature code covarage}.