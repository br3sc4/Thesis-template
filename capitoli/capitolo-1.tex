% !TEX encoding = UTF-8
% !TEX TS-program = pdflatex
% !TEX root = ../tesi.tex

%**************************************************************
\chapter{Introduzione}
\label{cap:introduzione}
%**************************************************************
In questo capitolo viene descritta l’azienda presso la quale si è svolto lo stage e le metodologie di lavoro utilizzate \
per i suoi progetti.

%**************************************************************
\section{L'azienda}

Wavelop è un’azienda giovane e innovativa, con sede operativa a Treviso e sede legale a Venezia, che si occupa dello \
sviluppo di applicazioni web e mobile per startup e imprese. L’ambiente è informale e dinamico, con un focus alla \
flessibilità e al lavoro da remoto. \

L’azienda riesce a soddisfare i requisiti dei clienti grazie all’utilizzo della metodologia Agile Scrum ed \
alla creazione di design user-centered, realizzando {mock-up} e prototipi per ottenere i risultati più adatti. \
Vengono utilizzate le migliori tecnologie in base al progetto richiesto, seguendo principi importanti \
come "mobile-first" e applicando i più efficienti design pattern.

%**************************************************************
\section{Tecnologie utilizzate}


%**************************************************************
\section{Metodologie di lavoro}

%**************************************************************
\section{Organizzazione del testo}


Riguardo la stesura del testo, relativamente al documento sono state adottate le seguenti convenzioni tipografiche:
\begin{itemize}
	\item gli acronimi, le abbreviazioni e i termini ambigui o di uso non comune menzionati vengono definiti nel glossario, situato alla fine del presente documento;
	\item per la prima occorrenza dei termini riportati nel glossario viene utilizzata la seguente nomenclatura: \emph{parola}\glsfirstoccur;
	\item i termini in lingua straniera o facenti parti del gergo tecnico sono evidenziati con il carattere \emph{corsivo}.
\end{itemize}