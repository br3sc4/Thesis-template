% !TEX encoding = UTF-8
% !TEX TS-program = pdflatex
% !TEX root = ../tesi.tex

%**************************************************************
\chapter{L'azienda}
\label{cap:azienda}
%**************************************************************
\section{Introduzione}

\vspace{20pt}
  \begin{figure}[!ht]
    \begin{center}
      \includegraphics[height=3cm, width=6cm]{logo-wavelop}
      \caption{Logo azienda Wavelop S.R.L.S}
      \textbf{Fonte:} \href{https://www.wavelop.com}{wavelop.com}
    \end{center}
  \end{figure}
\vspace{20pt} 

Wavelop è un’azienda giovane e innovativa, con sede operativa a Treviso e sede legale a Venezia, che si occupa dello \
sviluppo di applicazioni web e \emph{mobile} per \emph{startup} e imprese. L’ambiente è informale e dinamico, con un focus \
alla flessibilità e al lavoro da remoto. \\

L’azienda riesce a soddisfare i requisiti dei clienti grazie all’utilizzo della metodologia Agile Scrum ed \
alla creazione di \emph{design user-centered}, realizzando \emph{\gls{mock-up}} e prototipi per ottenere i risultati più adatti. \
Vengono, inoltre, utilizzate le migliori tecnologie in base al progetto richiesto.

%**************************************************************
\section{Tecnologie utilizzate}

\subsection{\emph{\Gls{backend}}}

\begin{itemize}
  \item \textbf{\emph{TypeScript}}: linguaggio basato su JavaScript, sviluppato da Microsoft, al quale viene aggiunta \
  la definizione statica dei tipi. Tutto ciò porta ad un rischio minore di \emph{bugs}.
  \item \textbf{\emph{Node.js}}: è un sistema \emph{runtime} per JavaScript orientato agli eventi asincroni, è progettato \
  per realizzare applicazioni di rete scalabili.
  \item \textbf{\emph{MongoDB}}: è un \emph{database} NoSQL \emph{document-oriented}, permette di sviluppare applicazioni \
  flessibili e che possono scalare facilmente.
  \item \textbf{Servizi \emph{\acrfull{aws}}}: sono i servizi \emph{cloud} forniti da Amazon che comprendono: \
  \emph{cloud-computing}, archiviazione, \emph{Machine-Learning} e \emph{Internet of Things}. L'azienda utilizza \
  principalmente: 
  \begin{itemize}
    \item AWS Lambda: computazione \emph{serverless};
    \item AWS DynamoDB: \emph{database} NoSQL;
    \item AWS API Gateway: creazione di \acrshort{api}.
  \end{itemize}
\end{itemize}

\subsection{\emph{\Gls{frontend}}}

\begin{itemize}
  \item \textbf{\emph{React}}: ...
  \item \textbf{\emph{Vue}}: ...
  \item \textbf{\emph{Angular}}: ...
\end{itemize}

%**************************************************************
\section{Processi aziendali}

\subsection{Metodologia Agile}
In questa sezione descrivo la metodologia Agile.

\subsection{Strumeti a supporto dei processi}
In questa sezione descrivo gli strumenti utilizzati dall'azienda per: la gestione del progetto, \emph{versioning}, \
\emph{issue tracking system} e per la documentazione.
